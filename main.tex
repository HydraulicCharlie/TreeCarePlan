\documentclass[12pt,a4paper]{article}
\usepackage[T1]{fontenc}
\usepackage[utf8]{inputenc}
\usepackage[swedish]{babel}
\usepackage{amsmath}
\usepackage{amsfonts}
\usepackage{amssymb}
\usepackage{lmodern}
\usepackage{hyperref}
\author{Christian Borg Bagge}
\usepackage{parskip}

\usepackage{geometry}
\usepackage{marginnote}

\title{Trädvårdsplan}
\date{2024-11-28}

\begin{document}

\maketitle

\section{Introduktion} 
\label{sec:introduktion}

Petter Krus håller i anförandet och välkomnar deltagarna.
140 deltagare, med ~60\% industri och ~40\% Akademiskt. 
Presentation Akademiskt ~70\% och ~30\% industri.


\subsection{Mål}

\subsubsection{Kultur-ansvar/-historiskt}

\subsubsection{Trivsel}
\subsubsection{Biologiska värden}
\subsubsection{Sociala Värden}
\subsubsection{Dränering/Dagvatten}
\subsubsection{Reducera buller}


\subsection{Ekonomiska}
\subsubsection{Renare luft}
\subsubsection{Skugga}


\subsection{Syfte}

\subsection{Genomförandeidé} 







\section{Indelning}
Pingstliljans samfällighetsförening består av fem olika områden namngivna efter gatunamnen: 



\begin{minipage}{0.5\textwidth}
\vspace{2mm}
\begin{itemize}
	\item Brudslöjan
	\item Ringblomman
	\item Pingstliljan
	\item Lupinen
	\item Påskliljan 
\end{itemize}
\end{minipage}%
\begin{minipage}{0.5\textwidth}
Vad kännetecknar området?

Vad vill vi ha för område?

\end{minipage}

\section{Avverkning}
\subsection{Påverkan biodepå}
\subsection{Avvägning mellan boende och samfällighet}
Försiktighet, Återhållsamhet och utredning. 

\section{Vårdplan}
Beskrivning av tidsaspekten.
Foto för att bestämma vilka träd som är. 
Foto för att beskriva åtgärder. 

\section{Omfattning}
\subsection{Föryngring}
\subsection{Särskilt skyddsvärda värden}

\end{document}
